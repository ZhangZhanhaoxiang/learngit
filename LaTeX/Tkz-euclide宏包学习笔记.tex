\documentclass[UTF8]{ctexart}
\title{Tkz-euclide宏包学习笔记}
\author{M81}
\date{\today}
\usepackage{tikz}
\usepackage{tkz-euclide}
\usepackage{listings}
\begin{document}
\maketitle
\tableofcontents
\section{宏包导入}
\begin{lstlisting}
\usepackage{tkz-euclide}
\begin{tikzpicture}
\end{tikzpicture}
\end{lstlisting}
\section{基本作图命令}
\subsection{点}
\subsubsection{定义}
\begin{lstlisting}
\tkzDefPoint(x,y){name}
\tkzDefPoint(a:d){name}
\tkzDefPoints{x1/y1/name1,x2/y2/name2,...}
\end{lstlisting}
\subsubsection{绘制}
\begin{lstlisting}
\tkzDrawPoint[option](P1)
\tkzDrawPoints[option](P1,P2,...)
\end{lstlisting}
P1、P2等点需在之前已经定义。
\subsubsection{标记}
\begin{lstlisting}
\tkzLabelPoint[option](point){$name$}
\end{lstlisting}
point:定义好的点  name:想要起的名字\\
option:方向:\\
\begin{tabular}{|c|c|c|}
\hline
above left & above & above right\\
\hline
left & point & right\\
\hline
below left & below & below right\\
\hline
\end{tabular}
\begin{lstlisting}
\tkzLabelPoints[option](P1,P2,...)
\end{lstlisting}
多个点同时标记,直接根据点的名字自动标记,默认标在右下角。
\subsection{直线}
\begin{lstlisting}
\tkzDrawLine[option](P1,P2)
\tkzDrawLines[option](P1,P2 P3,P4 P5,P6 ...)
\end{lstlisting}
option:
\begin{itemize}
    \item dashed - 虚线
\end{itemize}
\subsection{线段}
\begin{lstlisting}
\tkzDrawSegment[option](P1,P2)
\tkzDrawSegments[option](P1,P2 P3,P4 P5,P6 ...)
\end{lstlisting}
\subsection{圆}
\begin{lstlisting}
\tkzDrawCircle[option](A,B)
\end{lstlisting}
以A为圆心,B为圆上一点作圆。
\begin{lstlisting}
\tkzDrawCircle[option](A,B,C)
\end{lstlisting}
作关于A,B,C三点的圆。\\
option:
\begin{itemize}
    \item circum - 作过A,B,C三点的圆
    \item in - 作$\triangle ABC$的内切圆
\end{itemize}
\begin{lstlisting}
\tkzDrawCircle[option,R](A,r)
\end{lstlisting}
以A为圆心,r为半径作圆。
\begin{lstlisting}
\tkzDrawCircle[option,with nodes](A,B,C)
\end{lstlisting}
以A为圆心,BC为半径作圆。
\section{复杂作图命令}
\subsection{获取生成的点和线段的长度}
\begin{lstlisting}
\tkzGetPoint{P}
\tkzGetPoints{A}{B}
\tkzGetFirstPoint{A}
\tkzGetSecondPoint{B}
\tkzGetLength{a}
\end{lstlisting}
获取生成的点,并将其存储为点P;\\
获取同时生成的两个点;\\
获取同时生成的两个点中的第一个;\\
获取同时生成的两个点中的第二个;\\
获取返回的长度,并以pt为单位存储在变量a中,使用时可以用
\begin{lstlisting}
\a pt
\end{lstlisting}
来使用。
\subsection{特殊点}
\subsubsection{相交点}
\begin{lstlisting}
\tkzInterLL(A,B)(C,D)
\end{lstlisting}
返回直线AB与直线CD的交点。
\begin{lstlisting}
\tkzInterLC(A,B)(C,D)
\end{lstlisting}
返回直线AB与以C为圆心,经过点D的圆的交点。
\begin{lstlisting}
\tkzInterCC(A,B)(C,D)
\end{lstlisting}
返回以A为圆心,经过点B的圆和以C为圆心,经过点D的圆的交点。\\
后两者可以加上选项R或with nodes。
\subsubsection{中点}
\begin{lstlisting}
\tkzDefMidPoint(A,B)
\end{lstlisting}
返回线段AB的中点。
\subsubsection{任意点}
\begin{lstlisting}
\tkzDefPointOnline[pos=k](A,B)
\end{lstlisting}
以A为原点在线段AB上取一点,使$\bar{AP}=k\bar{AB}$,返回该点。
\begin{lstlisting}
\tkzDefPointOnCircle[angle=a,center=O,radius=r]
\end{lstlisting}
在以O为圆心,r为半径的圆上取一点P,使P与x轴的夹角为a,返回该点。\\
当圆是在前一行定义时,不需指定center和radius选项。
\subsection{特殊直线和圆}
\subsubsection{特殊直线}
\begin{lstlisting}
\tkzDefLine[option](A,B)
\tkzDefLine[option](A,B,C)
\end{lstlisting}
定义与若干个点有关的线段,返回该直线上的点。\\
option:
\begin{itemize}
    \item \textbf{与两个点有关:}
    \item mediator(默认) - 线段AB的垂直平分线,返回垂直平分线上的两个点
    \item perpendicular = through P(或orthogonal = through P) - 过点P且垂直于AB的直线,返回该垂线上除P外的一个点
    \item parallel = through P - 过点P且平行于AB的直线,返回该平行线上除P外的一个点
    \item \textbf{与三个点有关:}
    \item bisector [normed/out] - $\angle ABC$的角平分线,normed为内角平分线,out为外角平分线,返回该角平分线上除P外的一个点
\end{itemize}
定义完直线后,要用获取生成点的命令获取用于作图的辅助点,然后再用画直线命令来绘制。
\subsubsection{特殊圆}
\begin{lstlisting}
\tkzDefCircle[option](A,B)
\tkzDefCircle[option](A,B,C)
\end{lstlisting}
定义与若干个点有关的圆,返回该圆的圆心和半径。\\
option:
\begin{itemize}
    \item \textbf{与两个点有关:}
    \item through(默认) - 以A为圆心,B为圆上一点的圆
    \item diameter - 以AB为直径的圆
%    \item apollonius - 阿波罗尼圆,须在选项中指定比例K
%    \item orthogonal through = X and Y - 过X、Y且与以O为圆心、A为半径的圆正交的一个圆
%    \item orthogonal from = X - 以X为圆心且与以O为圆心、A为半径的圆正交的一个圆
    \item \textbf{与三个点($\triangle ABC$)有关:}
    \item circum - 外接圆
    \item in - 内切圆
    \item ex - (在当前字母顺序下)点B所对的旁切圆
    \item euler - 九点圆,也可写作nine
%    \item spieker - 斯俾克圆
\end{itemize}
定义完圆后,要用获取生成点和线段长度的命令获取用于作图的辅助点和半径,然后再用画圆命令来绘制。
\subsubsection{圆的切线}
\begin{lstlisting}
\tkzDefTangent[option](A,B)
\tkzDefTangent[option](A,r)
\end{lstlisting}
定义以A为圆心、B为其上一点(或r为其半径)的圆的切线。\\
option:
\begin{itemize}
    \item at = P - 过圆上一点P作圆的切线,返回切线上另一点
    \item from = P - 过圆外一点P作圆的切线(前者),返回两个切点
    \item from with R = P - 过圆外一点P作圆的切线(后者),返回两个切点
\end{itemize}
\subsection{特殊多边形}
\begin{lstlisting}
\tkzDefTriangle[option](A,B)
\end{lstlisting}
定义以AB为一边的特殊三角形,返回未定义的顶点。\\
option:
\begin{itemize}
    \item two angles = a1 and a2 - 以AB为底边,a1、a2为底角的三角形
    \item equilateral - 以AB为边的等边三角形
    \item pythagore - 三边长为3、4、5的直角三角形
    \item gold - 以AB为底边的黄金三角形
\end{itemize}
\begin{lstlisting}
\tkzDefRegPolygon[option,sides=n,name=P](A,B)
\end{lstlisting}
定义以AB为一边的正多边形,返回未定义的顶点。\\
option:
\begin{itemize}
    \item center - 点A为正多边形中心,B为其上一点
    \item sides - 边数
    \item name - 各顶点名称前缀,此例中各顶点名称为$P_1$, $P_2$, ..., $P_n$
\end{itemize}
\subsection{几何变换}
\begin{lstlisting}
\tkzDefPointBy[option](A)
\end{lstlisting}
返回A关于某一几何变换后的像。\\
option:
\begin{itemize}
	\item translation = from P to Q - 沿向量PQ平移
	\item homothety = center O ratio k - 以O为中心、位似比为k的位似变换
	\item reflection = over P--Q - 关于PQ作对称
	\item symmetry = center O - 关于O中心对称
	\item projection = onto P--Q - 在PQ上的射影
	\item rotation = center O angle a - 关于中心O作旋转角为a的旋转变换
	\item rotation in rad = center O angle a - 同上
	\item inversion = center O through P - 作关于以O为圆心、过P的圆的反演
\end{itemize}
\end{document}